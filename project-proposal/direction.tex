\documentclass[12pt, draft]{article}

\usepackage[margin=2cm]{geometry}
\usepackage{mathpartir}
\usepackage{lastpage}
% [final] because with draft, listings don't work
\usepackage[final]{listings}

\title{Toy Structurally Typed Language}
\author{Samuel Gr\"utter}
\date{}

% cmtt supports < and >
\usepackage{cmtt}

% plain text
\lstset{language=}

%%%%%%%%%%%%%%%%%%%%%%%%%%%%%%%%%%%%%%%%%%%%%%%%%%%%%%%%%%%%%%%%%%%%%%%%%%%%%%%%

\begin{document}

\parindent=0pt

\section{Direction}

The following points describe in which direction I would like to go and why I 
find this direction interesting.

-- draft -- TODO --

\begin{description}

  \item[Structural typing considered useful for practical programming] 
  Two examples:
  \begin{itemize}
  \item File exchange protocol (e.g. BitTorrent), message classes. 
        Crypto-extension to basic library without modifying basic libarary \
        and no boiler plate code.
  \item Combine two libraries with objects of same structure, but different
        "classes"
  \end{itemize}
  
  \item[Types behave as mathematical sets] 
  Easier than lattice and glb/lub.
  \begin{itemize}
     \item for the programmer (language user): tell ``just draw a Venn diagram''
     \item for the language spec
  \end{itemize} 
  
  \item[Simulate nominal typing with structural typing] 
  but the inverse does not work
  
  \item[Keep it simple and doable] 
  and a short spec is appealing
  
  \item[Clear separation of interfaces and implementation] 
  For instance, I believe that the lack of this separation is the source 
  of the problem described in section 6.2.2 of the DOT paper.

\end{description}

\section{Types and sets}

For each type \lstinline!T!, define the \emph{set of objects} as

$$oset(T) = \{ obj | obj : T \}$$

For each interface type \lstinline!I = [a1: T1, ..., aN: TN]!, define the 
\emph{set of properties} as 

$$pset(I) = \{ (a1: T1), \dots , (aN: TN) \}$$

When discussing type systems, it is important to specify whether we consider
types as sets of objects or as sets of properties. Both points of view are
interesting.

\end{document}
